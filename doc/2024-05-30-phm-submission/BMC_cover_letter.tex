\documentclass[a4paper,12pt]{letter}

% Adjust margins
\usepackage[left=2.5cm, right=2.5cm, top=2.5cm, bottom=2.5cm]{geometry}

% Opening and closing signatures
\signature{Ricarda Duerst\\Max Planck Institute for Demographic Research}
\date{\today}
\usepackage{hyperref}
\begin{document}

\begin{letter}{Editorial Board\\Population Health Metrics\\}

\opening{Dear Editors,}

Please find enclosed our manuscript, "Empirical prediction intervals applied to short term mortality forecasts and excess deaths" by Ricarda Duerst and Jonas Schöley, which we would like to submit for publication as an article in the Collection "COVID-19 and Impact on Mortality and Population Health" in Population Health Metrics.

During the Winter of 2022/23, Germany experienced an estimated 10\% increase in excess deaths, raising concerns about the reliability of such estimates given the inherent errors in statistical forecasting. Traditional parametric prediction intervals, commonly used in demographic forecasting, often fail to account for uncertainty from sources like model miss-specification or randomness and sudden shifts in model parameters, resulting in intervals that are too narrow. This leads to inaccuracies in assessing the true impact of the COVID-19 pandemic on population mortality, especially with the increased uncertainty in forecasts as more time has passed since the onset of the pandemic.

Our study aims to address the reliability of estimating a 10\% deviation in excess deaths by analyzing the error distribution in weekly death forecasts and the seasonality of death count fluctuations. We propose the use of empirical prediction intervals, derived from past forecasting errors, to provide more accurate and generalizable intervals compared to conventional parametric methods. These empirical intervals are then employed to quantify the probability of observing at least a 10\% excess death in various countries, allowing for a more reliable assessment of whether such increases are statistically significant and should be cause for concern.

By demonstrating the construction and evaluating the performance of empirical prediction intervals, our study aims to improve the reliability of excess death estimates and provide better tools for understanding mortality trends during pandemics.

We confirm that this manuscript has not been published elsewhere and is not under consideration by another journal. All authors have approved the manuscript and agree with its submission to Population Health Metrics. Ricarda Duerst declares no conflicts of interest. We would like to point out that Jonas Schöley is one of the editors of the Collection "COVID-19 and Impact on Mortality and Population Health".

To assist with the review process, we would like to suggest the following potential reviewers: Nico Keilman (\href{mailto:nico.keilman@econ.uio.no}{\texttt{nico.keilman@econ.uio.no}}), who is well-known for his work on empirical prediction intervals, and Johannes Bracher (\href{mailto:johannes.bracher@kit.edu}{\texttt{johannes.bracher@kit.edu}}), José Manuel Aburto (\href{mailto:jose-manuel.aburto@sociology.ox.ac.uk}{\texttt{jose-manuel.aburto@sociology.ox.ac.uk}}) and Silvia Rizzi \linebreak (\href{mailto:srizzi@health.sdu.dk}{\texttt{srizzi@health.sdu.dk}}), who all have extensive research experience in excess death and mortality modeling. Thank you for considering these suggestions. 

Sincerely,

Ricarda Duerst\\
Max Planck Institute for Demographic Research

\end{letter}

\end{document}
